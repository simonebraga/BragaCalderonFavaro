\documentclass[./main.tex]{subfiles}

\begin{document}

\subsection{Purpose}

\textbf{SafeStreets} is a crowd-sourced application that intends to
provide users with the possibility to notify authorities when traffic
violations occur. The main target of the application are violations that
can be easily captured by a camera (like, for instance, parking
violations). SafeStreets intends also to provide users with the
possibility to mine the stored information with different levels of
visibility. Moreover, the application must cross the collected data with
information coming from the municipality to provide suggestions on
possible interventions to decrease the incidence of violations and
accidents. In the end, the application must forward data about
violations to generate traffic tickets, and must allow authorities to
get statistics on issued tickets.

\subsection{Scope}

The services of SafeStreets must be developed through a mobile application.
Everyone can use the mobile application, whose behavior is different depending
on the user's privileges. Excluding the secondary actors (that interact
directly with the back-end of the application with the purpose to exploit the
front-end services), 3 types of user can be identified:

\begin{itemize}
	\item Common user
	\item Authority
	\item Municipality user
\end{itemize}

SafeStreets must provide common users with the possibility to notify a traffic
violation by taking a picture. Then the data about the violation is sent to the
application server that checks its integrity before storing it in a database.
Common users can access the data stored in the database with some restrictions.
SafeStreets provides them with the possibility to select some filters for
querying the data, and with a proper graphic interface to visualize the queried
data.\medskip\\
Authorities are provided with the possibility to query without restrictions the
data stored by SafeStreets. The application server must use a service
responsible for the generation of traffic tickets. The results of the ticket
generations are stored by SafeStreets and analyzed with the purpose to provide
in-depth statistics. To do so, SafeStreets relies on a data warehousing system
that gets the information from the primary database and analyzes it with data
mining techniques. Authorities must be provided with the proper interfaces for
this particular type of query.\medskip\\
Municipality users must be provided with the possibility to get suggestions on
possible interventions to reduce the incidence of accidents and violations. To
do so, SafeStreets relies on the data warehousing system described so far, that
must apply data analysis techniques to identify possible solutions. SafeStreets
can access information about accidents provided by the municipality to identify
more precise interventions. Data from the municipality must be periodically
checked for updates.

\subsection{Definitions and acronyms}

\begin{itemize}
\item
  \textbf{User}\\
  The consumer of the application. It includes common users, authorities
  and municipality users.
\item
  \textbf{Common user}\\
  The user type that everyone can sign up as. It does not require any
  kind of verification.
\item
  \textbf{Authority}\\
  The user type that authorities can get. It requires the verification
  of an activation code.
\item
  \textbf{Municipality user}\\
  The user type that municipal employees can get. It requires the
  verification of an activation code.
\item
  \textbf{Municipality Tickets Service (MTS)}\\
  Service offered by the municipality to generate traffic tickets from
  information about the violations.
\item
  \textbf{Optical Character Recognition (OCR)}\\
  Software that converts text scanned from a photo in a machine-encoded
  text.
\item
  \textbf{Query interface}\\
  The interface provided to the users to select some filters when
  requesting data.
\item
  \textbf{Application Programming Interface (API)}\\
  An interface or communication protocol between client and server
  intended to simplify the building of client-side software.
\item
  \textbf{Extract Transform Load (ETL)}\\
  The general procedure of copying data from one or more sources into a
  destination system which represents the data differently from the
  sources or in a different context than the sources.
\item
  \textbf{User Experience (UX)}\\
  The person's emotions and attitudes about using a particular product, system
  or service.
\end{itemize}

\subsection{Revision history}

\begin{table}[H]
\centering
\begin{tabular}{|c|c|c|}
\hline
Version & Release date & Description\tabularnewline
\hline
1.0 & December 9, 2019 & First release\tabularnewline
%TODO Add new information for every new release
\hline
\end{tabular}
\end{table}

\subsection{Document Structure}

\textbf{Section \ref{introduction}} contains a summary of the features of the
application and a description of the purpose of this document. The description
is slightly different from the one provided in the RASD, as it captures a more
practical point of view of the application. It also contains all the necessary
elements for the comprehension of this document.
\medskip\\
\textbf{Section \ref{architectural_design}} is the core of the document. It
contains all the most important notions for the developing process of the
application. It includes the description of the design architecture and the
analysis of the chosen design patterns. Every subsection offers a different
point of view of the application: in this sense, it can be considered as a
"framework" where it is possible to access information with different levels of
granularity.
\medskip\\
\textbf{Section \ref{user_interface_design}} contains a large set of mockups
that capture all the important aspects of the communication between the user
and the application. Moreover, this section contains a formal description of
the UX flow, which was not included in the previous section because it is not
strictly part of the system architecture.
\medskip\\
\textbf{Section \ref{requirements_traceability}} contains a formal description
of the mapping between the requirements and the components responsible for
their exploitation.
\medskip\\
\textbf{Section \ref{implementation_integration_test_plan}} contains a
description of how the project should be developed. It is a useful reference
for the optimization of the developing process as it contains a forecast on how
much the developing of every component is onerous. It also contains a plan for
the testing and integration process.
\medskip\\
\textbf{Section \ref{effort_spent}} includes information about the number of
hours each group member has worked for this document.
\medskip\\
\textbf{Section \ref{references}} includes the references to the tools used to
draw up this document.

\end{document}